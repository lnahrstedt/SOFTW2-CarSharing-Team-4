Das System besteht aus einem Softwaresystem und der dazugehörigen Hardware,
einschließlich der Dokumentation.
Grundlegend soll das System eine automatische Abwicklung der Buchungen,
Verwaltung der Fahrzeuge, Mitglieder und Abrechnungen ermöglichen.
Es ist darauf ausgelegt, eine hohe Erreichbarkeit zu gewährleisten.
Dafür bietet es Ihren Kunden verschiedene Möglichkeiten, ein Mitgliedskonto anzulegen oder zu verwalten:
entweder direkt in der Filiale, telefonisch oder online. \medskip

Darüber hinaus verfügt das System über eine benutzerfreundliche Web-Schnittstelle,
welche für alle Nutzer den Zugang ermöglicht.
Innerhalb der Web-Schnittstelle kann sich der Nutzer anmelden,
sofern er bereits über ein Konto verfügt.
Sobald die Anmeldung durchgeführt wurde, wird der Gast als Mitglied, Mitarbeiter oder Admin
klassifiziert und erhält die ihm zustehenden Rechte und Funktionalitäten.
Darüber wird beispielsweise den Mitarbeitern ermöglicht, den Fahrzeugpool, die Ausleihstationen und die
Abrechnungen zu verwalten.
Somit ist der einzige Berührungspunkt aller Nutzer mit dem System die Web-Schnittstelle.
Dabei wird ein besonderes Augenmerk auf den Schutz der Kunden- und Firmendaten gelegt,
indem alle übermittelten Daten verschlüsselt werden. \medskip

Um Ihren Kunden ein bestmögliches Nutzungserlebnis zu bieten, benötigt das System weitere Schnittstellen.
Eine solche Schnittstelle ermöglicht es beispielsweise Ihren Kunden,
reservierte Fahrzeuge mithilfe einer RFID-Karte oder eines vergleichbaren Systems zu nutzen. \medskip

Ein besonderes Alleinstellungsmerkmal des Systems ist die Anbindung
an das bestehende Buchhaltungssystem Ihres Unternehmens.
Darüber hinaus soll das System ausfallsicher sein,
mehrere Sprachen unterstützten und idealerweise automatisierte Workflows vorweisen. \medskip

Da Sie davon ausgehen, dass es ca.\ 5000 Mitglieder, 25 Fahrzeuge und 5 Stationen pro Standort
bei einer Anzahl von 10 Standorten geben wird, muss bei der Entwicklung des Systems auf die Robustheit
sowie die Leistungs-/ Performanceanforderungen geachtet werden.
Das zu liefernde Softwaresystem besteht aus einem Frontend, Backend und einer Datenbank,
auf der u.a.\ Kundendaten gespeichert und verwaltet werden.
Die Hardware umfasst RFID-Karten, RFID-Lesegeräte und GPS-Tracker.
Die Dokumentation des Projekts enthält das Pflichtenheft, die technische Dokumentation,
das Product Backlog der Zwischenergebnisse und einen Projektplan.