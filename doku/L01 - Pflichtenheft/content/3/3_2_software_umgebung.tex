Da im Unternehmen bereits mehrere Linux-Server vorhanden sind, sollte das Backend mit dem Linux-Betriebssystem
kompatibel sein.
Dies wird mit der Hilfe der Containervirtualisierungssoftware Docker realisiert.
Diese kann auf den Servern installiert werden.
Da auf den PCs des Unternehmens Windows installiert ist, sollte das Frontend, was von den Mitarbeitern bedient wird
mit Windows kompatibel sein.
Dies wird erreicht, indem diese Schnittstelle in Form einer Webanwendung zur Verfügung gestellt wird.
Ferner ist es geplant den Dienst auch als App anzubieten, die dieselben Funktionen beinhaltet.