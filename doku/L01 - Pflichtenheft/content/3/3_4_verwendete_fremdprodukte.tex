Das Backend wird mit Java und Spring Boot entwickelt werden, da wir mit der
genannten Software bereits Erfahrung gesammelt haben und die beinhalteten Funktionen
die Entwicklung erleichtern.
Für die Entwicklung des Frontends haben wir uns für Flutter entschieden,
wobei es sich um ein UI-Entwicklungs-Kit von Google handelt.
Dieses ist zusätzlich Open-Source und es können Plattformunabhängige Apps, sowie Browseranwendungen in der
Programmiersprache Dart erstellt werden.
Nach der Registrierung soll eine Identitätsprüfung stattfinden.
Hierfür wird das Postident-Verfahren verwendet.
Des Weiteren soll eine Bonitätsprüfung stattfinden, was über eine SCHUFA Abfrage realisiert wird.
Die Abwicklung der Zahlungen wird von einem externen Dienstleister übernommen.
Dabei sollen die Zahlungen über Paypal, SEPA-Lastschrift, Kreditkarte und Überweisung möglich sein.