Als Auftragnehmer liegt es in unserer Verantwortung, dass die Anwendung alle relevanten rechtlichen Randbedingungen einhält.
Dazu zählen: \medskip
\begin{itemize}
    \item Datenschutz: Die Verarbeitung personenbezogener Daten wie Name, Adresse oder Zahlungsinformationen durch die Anwendung entspricht den Datenschutzbestimmungen gemäß der Datenschutz-Grundverordnung (DSGVO).
    \item Haftung: Die Anwendung stellt die Haftungsregularien des Auftraggebers dar, die im Falle eines Unfalls oder Schadens gelten.
    \item Urheberrecht: Bei der Verwendung von Bildern oder Markennamen in der Anwendung wird sichergestellt, dass keine Urheberrechte verletzt werden.
    \item Verbraucherschutzrecht: Die Anwendung stellt sicher, dass die User angemessen über die Nutzungsbedingungen, die Preise und die Vertragsbedingungen des Auftraggebers informiert werden.
\end{itemize}

Der Bundestagsbeschluss zur Bevorrechtigung des Carsharings hat keine direkte Auswirkung auf die Carsharing-Anwendung.
Gemäß dem Beschluss soll das Konzept des Carsharing gefördert und unterstützt werden, indem bevorzugte Parkplätze oder Zufahrtswege für Carsharing-Fahrzeuge bereitgestellt werden.
Der Bundestagsbeschluss zur Bevorrechtigung des Carsharings hat keine direkte Auswirkung auf die Carsharing-Anwendung.