Der abstrakte Akteur \emph{Benutzer} hat in der Anwendung die Möglichkeit, seine bevorzugte Sprache auszuwählen.
Somit haben alle weiteren Akteure ebenfalls die Möglichkeit dies zu tun.
Dies ist in UC-3 wiedererkennbar. \medskip

Beim Besuchen der Web-Anwendung haben alle Benutzer die Rolle eines Gastes inne.
Der Akteur \emph{Gast} kann sich einloggen (UC1).
An dem Login ist der Authentifizierungsdienstleister beteiligt.
Des Weiteren kann ein Gast ein Mitgliedskonto anlegen (UC4).
Dabei muss er einen der angebotenen Tarife auswählen (UC6).
Außerdem ist hier die SCHUFA beteiligt, da über sie eine Bonitätsprüfung vorgenommen wird.
Sofern es sich bei der erstellenden Person um ein Mitarbeiter handelt, kann ein Admin im Nachgang das Mitgliedskonto
in ein Mitarbeitermitgliedskonto umwandeln (UC23).
Dies ist im Diagramm durch den extension point \enquote{Mitarbeiter sein} beschrieben.
Ein Mitgliedskonto kann vor Ort, in einer Filiale oder vom Gast selbst in der Web-Anwendung angelegt werden.
Aufgrund der Ähnlichkeit der Prozesse, aber der Unterschiede in der Durchführung, haben wir uns für die Verwendung von Vererbung entschieden.
Aus diesem Grund erben die Anwendungsfälle in Filiale anlegen (UC7), telefonisch anlegen (UC8) und Online anlegen (UC5) von \enquote{Mitgliedskonto anlegen}.
Wird ein Konto vom \emph{Gast} selbst angelegt, so ist der \emph{Identifizierungsdienstleister} an dem Anwendungsfall beteiligt.
Wird das Mitgliedskonto in einer Filiale angelegt, ist ein \emph{Mitarbeiter} an dem Anwendungsfall beteiligt.
Dieser übernimmt auch die Identitätsprüfung. \medskip

Der Akteur \emph{Registrierter Benutzer} unterscheidet sich von dem \emph{Gast}, indem er bereits registriert und angemeldet ist.
Somit wird ein \emph{Gast} zu einem \emph{Registrierten Benutzer}, sobald er sich eingeloggt oder registriert hat.
Zudem kann er jederzeit einen anderen Tarif wählen.
Ein \emph{Registrierter Benutzer} kann sich außerdem ausloggen, wodurch er wieder zu einem Gast wird.
Daran ist wiederum ein Authentifizierungsdienstleister beteiligt.
Weiterhin kann er eine Buchung verwalten, dazu gehört es, eine Buchung einzusehen, eine neue Buchung zu machen
und eine Buchung zu stornieren.
Wenn eine Buchung storniert werden soll, der Tarif aber Stornierungskosten nicht abdeckt, so muss der Benutzer die
Stornogebühr bezahlen.
Bevor das \emph{Mitglied} eine neue Buchung erstellen (UC13) oder seine Mitgliedschaft kündigen (UC15) kann, muss es offene Rechnungen bezahlen (UC18).
Die soeben genannten Fälle haben wir in dem Diagramm durch einen extension Point visualisiert.
Beim Bezahlen ist der \emph{Zahlungsdienstleister} beteiligt.
Zudem kann der \emph{Registrierte Benutzer} Mitgliedskontodaten verwalten (UC9) und Abrechnungen einsehen (UC12).
Schließlich kann ein \emph{Registrierter Benutzer} im Umkreis nach verfügbaren Fahrzeugen suchen (UC10). \medskip

Das \emph{Mitglied} kann eine reservierte Fahrt antreten (UC16) und diese abschließen (UC17).
Des Weiteren kann das \emph{Mitglied} seine offenen Rechnungen bezahlen (UC18). \medskip

Der \emph{Mitarbeiter} kann ein Fahrzeug verwalten (UC19).
Darüber hinaus kann ein \emph{Mitarbeiter} Ausleihstationen verwalten (UC20).
Ferner kann dieser den Fuhrpark verwalten (UC21).
Zuletzt kann ein Mitarbeiter Fahrzeuginformationen erfassen (UC22). \medskip

Der Akteur \emph{Admin} kann ein Mitgliedskonto zu einem Mitarbeitermitgliedskonto umwandeln (UC23)
sowie Mitarbeiterzugänge verwalten (UC24).