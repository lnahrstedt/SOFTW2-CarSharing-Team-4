Für die Lauffähigkeit der Software werden mehrere Schnittstellen benötigt, die dem System interne
und externe Funktionalitäten zur Verfügung stellen müssen.
Eine interne Schnittstelle muss zwischen dem System und dem älteren Buchhaltungssystem liegen, damit auf die
Buchungsdaten der Kunden zugegriffen werden kann.
Da das Buchhaltungssystem bereits vorhanden ist und somit bereits Ihre Anforderungen erfüllt, fallen für die
Erweiterung keine neuen Anforderungen an.
Zusätzlich müssen die Daten über die interne Schnittstelle zwischen Frontend und Backend verschlüsselt übertragen werden. \medskip

Aufgrund der Tatsache, dass wir für die Aspekte Zahlung, Identifikation, Authentifizierung und Bonitätsprüfung Drittunternehmen
beschäftigen, werden hierfür externe Schnittstellen benötigt.
Zusätzlich wird für die persistente Speicherung der Daten ein Datenbankdienstleister herangezogen.
Somit fallen für diese Schnittstellen besondere Anforderungen an:
\begin{itemize}
    \item Alle Dienstleister müssen sicherstellen, dass die übertragenden Daten nicht für Dritte einsehbar sind.
    \item Alle Dienstleister müssen sicherstellen, dass ihre Dienste immer zu den gegebenen Zeiten erreichbar sind.
    \item Der Identifikationsdienstleister muss sicherstellen,
    dass die Identität des Benutzers gemäß deutschen Richtlinien korrekt verifiziert wird.
    \item Der Authentifizierungsdienstleister muss sicherstellen, dass bei einem ein- und ausloggen aus dem System das
    korrekte Benutzerkonto geladen wird.
    \item Der Zahlungsdienstleister muss sicherstellen, dass die Zahlung gemäß deutschen Richtlinien vonstattengeht und
    dass die Zahlung des Kunden ordnungsgemäß weitergeleitet werden.
    \item Der Datenbankdienstleister muss benötigte Daten mit einer ausreichender geschwindigkeit zur Verfügung stellen.
\end{itemize} \medskip

Neben den hinzugezogenen Dienstleistern ist außerdem eine Schnittstelle für GPS-Tracker und Kartenleser vorgesehen.
Beide dieser Geräte werden in den Autos verbaut und benötigen einen Internetzugriff, um mit dem System kommunizieren
zu können.