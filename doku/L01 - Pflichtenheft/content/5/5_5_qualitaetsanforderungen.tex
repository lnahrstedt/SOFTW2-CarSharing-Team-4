\subsubsection{Zuverlässigkeitsanforderungen}
Nach Service Level Agreement ist eine Zuverlässigkeit von 99,5\% vereinbart.
Zusätzlich bieten wir ein 24/7 Supportmodell an, um eine höchstmögliche Verfügbarkeit zu garantieren.

\subsubsection{Leistungs-/Performanceanforderungen}
Leistungs-/ Performanceanforderungen gibt es seitens des Kunden nicht.
Allerdings ist es sinnvoll, dass die erwarteten 50.000 Mitglieder gleichzeitig ohne merkliche Verzögerungen auf
die Anwendung zugreifen können.

\subsubsection{Sicherheitsanforderungen}
Innerhalb der Anwendung müssen alle Daten verschlüsselt übermittelt werden.
Des Weiteren gibt es mindestanforderungen für Passwörter, um den Kunden zu schützen.
Ein Passwort muss aus mindestens acht Zeichen bestehen, dabei muss es mindestens einen Großbuchstaben, einen Kleinbuchstaben,
ein Sonderzeichen und eine Zahl enthalten.

\subsubsection{Kompatibilitäts-/Portabilitätsanforderungen}
Die Backend-Software muss auf dem bereits vorhandenen Linux Server laufen, dies wird durch die Verwendung
von Docker gewährleistet.
Zudem muss die Benutzeroberfläche auf den im Unternehmen verwendeten Windows PCs funktionieren.
Dies wird durch das Entwicklungskit \enquote{Flutter} gewährleistet, da es ermöglicht Plattformunabhängige Software,
ohne große Anpassungen zu erstellen.
Die erste Version wird als Webanwendung bereitgestellt, allerdings können ohne große Anpassungen vorzunehmen
auch Android oder iOS Anwendungen erstellt werden, um eine größere Portabilität zu gewährleisten.