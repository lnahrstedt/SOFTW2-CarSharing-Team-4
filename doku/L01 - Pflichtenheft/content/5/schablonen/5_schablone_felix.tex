\begin{table}[H]
    \centering
    \caption{Anwendungsfallbeschreibung zu: \emph{Bezahlen}}
    \label{tab:anwendungsfallbeschreibung_bezahlen}
    \begin{tabularx}{\textwidth}{|l|X|}
        \toprule
        \textbf{Anwendungsfall}           & [UC18]                             \\
        \hline
        \textbf{Name}                     & Bezahlen                           \\
        \hline
        \textbf{Initiierender Akteur}     & Mitglied                           \\
        \hline
        \textbf{Weitere Akteure}          & Zahlungsdienstleister              \\
        \hline
        \textbf{Kurzbeschreibung} & Das Mitglied begleicht eine offene Rechnung.
        Der Zahlungsprozess wird von einem Dienstleister abgewickelt. \\
        \hline
        \textbf{Vorbedingung}             & Es liegt eine offene Rechnung vor. \\
        \hline
        \textbf{Nachbedingung}            & Die offene Rechnung ist beglichen. \\
        \hline
        \textbf{Ablauf} &
        \begin{enumerate}
            \item Das Mitglied navigiert im User-Interface zur Zahlungsansicht.
            \item Das Mitglied wird an den Zahlungsdienstleister weitergeleitet, welcher die Zahlung abwickelt.
        \end{enumerate} \\
        \hline
        \textbf{Alternativen}             & N/A                                \\
        \hline
        \textbf{Ausnahmen}                & N/A                                \\
        \hline
        \textbf{Benutzte Anwendungsfälle} & N/A                                \\
        \hline
        \textbf{Datenanforderungen}       & Buchungsdaten                      \\
        \bottomrule
    \end{tabularx}
\end{table}