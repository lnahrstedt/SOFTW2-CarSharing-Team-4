\begin{table}[H]
    \centering
    \caption{Anwendungsfallbeschreibung zu: \emph{Mitgliedskonto anlegen}}
    \label{tab:anwendungsfallbeschreibung_mitgliedskonto_anlegen}
    \begin{tabularx}{\textwidth}{|l|X|}
        \toprule
        \textbf{Anwendungsfall}           &  [UC04]                            \\
        \hline
        \textbf{Name}                     & Mitgliedskonto anlegen                           \\
        \hline
        \textbf{Initiierender Akteur}     & Gast                           \\
        \hline
        \textbf{Weitere Akteure}          & Mitarbeiter oder Identifizierungsdienstleister, SCHUFA          \\
        \hline
        \textbf{Kurzbeschreibung} & ein Gast legt ein Konto vor Ort mit einem Mitarbeiter
        oder Online, bzw. telefonisch mithilfe von Dienstleistern an \\
        \hline
        \textbf{Vorbedingung}             & der Gast muss mindestens 18 Jahre alt sein \\
        \hline
        \textbf{Nachbedingung}            & der Gast ist ein Mitglied \\
        \hline
        \textbf{Ablauf} &
        \begin{itemize}
            \item Online
            \begin{enumerate}
                \item Der Gast gibt die Startseite des Carsharing-Unternehmens im Browser ein
                \item Auf der Homepage klickt der Gast auf ``Mitglied werden''
                \item Der Gast gibt seine Personaldaten ein und wird durch einen Identifizierungsdienstleister bestätigt
                \item Es folgt eine Bonitätsprüfung durch die SCHUFA
                \item Sind alle Daten korrekt wählt der Gast einen Tarif
            \end{enumerate}
        \end{itemize}\\
        \hline
        \textbf{Alternativen}             &
        \begin{itemize}
            \item In einer Filiale
            \begin{enumerate}
                \item Der Gast spricht einen Mitarbeiter bezüglich des Mitgliedskontos an
                \item Der Mitarbeiter legt dem Gast ein Mitgliedskonto an und trägt die Personalien ein
                \item Es folgt eine Bonitätsprüfung durch die SCHUFA
                \item Sind alle Daten korrekt wählt der Gast einen Tarif
                \item Der Gast legt ein Password fest
            \end{enumerate}
            \item Telefonisch
            \begin{enumerate}
                \item Der Gast spricht den Kundenservice bezüglich des Mitgliedskontos an
                \item der Mitarbeiter legt dem Gast ein Mitgliedskonto an und trägt die Personalien ein
                \item Es folgt eine Bonitätsprüfung durch die SCHUFA
                \item Sind alle Daten korrekt wählt der Gast einen Tarif aus
                \item dem Kunden wird ein Initialpasswort postalisch übermittelt
            \end{enumerate}
        \end{itemize}\\
        \bottomrule
    \end{tabularx}
\end{table}

\begin{tabularx}{\textwidth}{|l|X|}
\hline
\textbf{Ausnahmen}                & N/A                                \\
\hline
\textbf{Benutzte Anwendungsfälle} & Tarif wählen, in Filiale anlegen, nicht in Filiale anlegen           \\
\hline
\textbf{Datenanforderungen}       & Personalien, ggf. Führerschein \\
\bottomrule
\end{tabularx}