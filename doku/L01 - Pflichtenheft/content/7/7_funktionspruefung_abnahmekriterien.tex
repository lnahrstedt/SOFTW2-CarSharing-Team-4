Jedes Element Ihres Systems soll den höchsten Qualitätsstandards entsprechen und fehlerfrei funktionieren.
Hierfür werden geeignete Softwaretestverfahren durchgeführt.
Neben den Komponententests, die einzelne Elemente des Systems überprüfen, führen wir auch Integrationstests durch.
Unsere Teststrategie umfasst eine funktionsorientierte Integration, bei der die Module in Abhängigkeit von einzelnen Anwendungsfällen integriert werden.
Dies bedeutet, dass wir sicherstellen, dass alle Funktionen des Systems in ihrem Zusammenspiel reibungslos funktionieren und Ihren Anforderungen entsprechen.
Nach Abschluss der Implementierungsphase werden wir einen vollumfänglichen Systemtest durchführen.
Hierbei überprüfen wir nicht nur die funktionalen Anforderungen, sondern auch die nicht-funktionalen Anforderungen, sodass das System Ihren Bedürfnissen entspricht.
Des Weiteren legen wir großen Wert auf die Benutzererfahrung, damit Ihre Kunden optimal von der Nutzung profitieren können.
Am Ende des Entwicklungsprozesses führen wir gemeinsam mit Ihnen einen Abnahmetest durch.
Dieser ist der letzte Test vor Freigabe des Systems.
Dabei können Sie unter realen Bedingungen testen, ob das System alle Anforderungen erfüllt und es ihren Wünschen entspricht.

\bigskip

Um die vollständige Funktionsfähigkeit des Gesamtsystems zu gewährleisten, haben wir folgende Abnahmekriterien klar definiert:
\medskip
\begin{itemize}
    \item Alle Tests, die die Funktionsfähigkeit des gesamten Systems überprüfen, wurden vollständig fehlerfrei durchgeführt.
    \item Alle Mindestanforderungen, die gemeinsam mit Ihnen definiert wurden, müssen vollständig implementiert und funktionsfähig sein.
    \item Die Hardware muss vollständig eingerichtet und fehlerfrei in das Gesamtsystem integriert sein.
\end{itemize}

