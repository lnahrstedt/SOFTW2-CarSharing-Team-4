Die primären Akteure repräsentieren die Nutzer der Anwendung und damit die unterschiedlichen Rollen mit verschiedenen Zugriffsrechten und Nutzungsintentionen.
Von diesen haben wir sechs vorgesehen.
Diese sind: \emph{Gast}, \emph{Benutzer}, \emph{Registrierter Benutzer}, \emph{Mitglied}, \emph{Mitarbeiter} und \emph{Admin}.
Die primären Akteure in dem Diagramm unterliegen einer Vererbungshierarchie, die auf der Beziehung \enquote{ist-ein} basiert.
Das bedeutet, dass bestimmte Akteure aus anderen Akteuren abgeleitet werden.
Die höchste Stufe in der Hierarchie ist der abstrakte Akteur \emph{Benutzer}.
Direkt unterhalb des abstrakten Benutzers gibt es zwei weitere Akteure.
Den Gast und den ebenfalls abstrakten Akteur \emph{Registrierter Benutzer}.
Diese Akteure erben alle Eigenschaften des abstrakten Akteurs \emph{Benutzer} und haben jeweils zusätzliche Funktionen und Berechtigungen.
Aus dem abstrakten Akteur \emph{Registrierter Benutzer} leiten sich wiederum \emph{Mitglied} und \emph{Mitarbeiter} ab.
Als spezieller Typ des Akteurs \emph{Mitarbeiter} erbt der \emph{Admin} alle Eigenschaften und Funktionen von \emph{Mitarbeiter}.
Außerdem verfügt er über zusätzliche Berechtigungen und Funktionen, die es ihm ermöglichen, administrative Aufgaben auszuführen.

\medskip

Durch die Verwendung der Vererbung können gemeinsame Eigenschaften und Funktionen der Akteure effektiv verwaltet werden.
Dies erleichtert die Implementierung und Wartung der Anwendung.
Die Akteure \enquote{Benutzer} und \enquote{Registrierter Benutzer} sind abstrakt, weil sie in der Anwendung selbst nicht in Erscheinung treten, sondern ihre Erben Anwendungsfälle teilen und somit eine \enquote{Oder}-Beziehung dargestellt werden kann.
Durch die Verwendung abstrakter Akteure können wir außerdem gemeinsam genutzte Anwendungsfälle der Erben zusammenfassen, ohne redundante oder unnötige Informationen in dem Anwendungsfalldiagramm darzustellen.