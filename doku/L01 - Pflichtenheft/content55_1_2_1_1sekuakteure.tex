Die im Diagramm genannten sekundären Akteure wie der Zahlungsdienstleister, Identifizierungsdienstleister, Authentifizierungsdienstleister und die SCHUFA, bieten wichtige Schnittstellen für die Carsharing-Anwendung.
Der Zahlungsdienstleister ist für die Abwicklung der Zahlungen innerhalb der Anwendung zuständig und stellt sicher, dass alle Zahlungen sicher und zuverlässig abgewickelt werden.
Bei der Registrierung für den Carsharing-Dienst übernimmt der Identifizierungsdienstleister, zum Beispiel Postident, die Verantwortung für die Verifizierung der Identität der Benutzer.
Dazu prüft der Dienstleister die personenbezogenen Daten anhand eines Lichtbilddokuments, um sicherzustellen, dass sie korrekt und gültig sind.
Zusätzlich prüft dieser den Führerschein der sich registrierenden Personen, um sicherzustellen, dass diese berechtigt sind, Fahrzeuge zu führen.
Der Authentifizierungsdienstleister wiederum sorgt für die korrekte Authentifizierung der Benutzer beim Einloggen in die Anwendung.
Dadurch wird verhindert, dass unbefugte Personen Zugang zu den Daten und Funktionen der Anwendung haben.
Die SCHUFA ist für die Prüfung der Kreditwürdigkeit der Benutzer zuständig.
Sie stellt sicher, dass nur Benutzer mit einer guten Bonität Zugang zur Anwendung haben und dass mögliche Zahlungsverzugsrisiken minimiert werden.

Ein weiterer zentraler Akteur ist der Datenbankdienstleister.
Dieser ist dafür verantwortlich, ein sicheres und effizientes Datenbanksystem bereitzustellen.
Das Datenbanksystem speichert alle relevanten Informationen, wie z.\ B.\ Benutzerdaten, Fahrzeuginformationen, Buchungen und Abrechnungen.
Der Datenbankdienstleister ist entgegen allen zuvor genannten Akteuren nicht im Diagramm aufgeführt, da er an jeder Stelle in der Anwendung involviert ist, wo Daten verarbeitet werden.
Eine explizite Darstellung würde die Übersichtlichkeit des Diagramms beeinträchtigen.