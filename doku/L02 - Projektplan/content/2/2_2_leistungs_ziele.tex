Die Leistungsziele lassen sich in vier Kategorien aufteilen, für die jeweils gesonderte Anforderungen
gestellt werden können:

\begin{itemize}
    \item \textbf{Geschwindigkeit:}
    \newline Die Anwendung soll eine reaktive Benutzeroberfläche bieten,
    um Kunden und Mitarbeiter ein zufriedenstellendes Nutzererlebnis zu bieten.
    \item \textbf{Kapazität:}
    \newline Die Anwendung soll genügend Kapazität bieten, um die von dem Kunden geplante Nutzeranzahl
    zu unterstützen.
    In dem System soll es bei hoher Auslastung nicht zu erheblichen Verlangsamungen kommen.
    \item \textbf{Zuverlässigkeit:}
    \newline Die Anwendung soll eine möglichst geringe Ausfallrate haben, damit die Nutzer jederzeit mit dem
    System interagieren können.
    Das im Pflichtenheft festgelegte Service Level Agreement von 99,5\% soll eingehalten werden.
    \item \textbf{Skalierbarkeit:}
    \newline Die Anwendung soll in der Lage sein, ohne größeren Aufwand auf eine steigende Nutzerzahl angepasst zu werden.
\end{itemize}