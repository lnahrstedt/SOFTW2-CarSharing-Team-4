Terminliche Ziele helfen bei der Einhaltung des Plans.
Die wichtigsten Teile des Projekts so wie die von dem Kunden geforderten Abgaben werden einem Termin zugwiesen.
So kann der Verlauf zu jeder Zeit kontrolliert werden. \medskip

Darüber hinaus ermöglichen terminliche Ziele eine effektive Ressourcenplanung und -zuweisung.
Indem jeder Aufgabe ein bestimmter Zeitrahmen zugewiesen wird, kann besser abgeschätzt werden,
welche Ressourcen, sei es Arbeitskräfte, finanzielle Mittel oder Materialien, zu welchem Zeitpunkt benötigt werden.
Dies erleichtert die Organisation des Projektablaufs und minimiert Engpässe oder Verzögerungen. \medskip

Terminliche Ziele helfen auch bei der Priorisierung von Aufgaben.
Durch die Festlegung von klaren Fristen können Teammitglieder und Stakeholder ihre Aktivitäten entsprechend
planen und wissen, welche Aufgaben priorisiert werden müssen, um die gesetzten Termine einzuhalten.
Dies fördert eine effiziente Zusammenarbeit und sorgt für einen reibungslosen Ablauf des Projekts. \medskip

Die von uns gesetzten Termine wurden in Form von Meilensteinen niedergeschrieben und sind in
\autoref{sec:meilensteine} zu sehen.