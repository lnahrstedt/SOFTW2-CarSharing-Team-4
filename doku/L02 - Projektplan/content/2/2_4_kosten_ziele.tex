Für die Planung des Projekts ist es von großer Bedeutung, dass die entstehenden Kosten berücksichtigt werden.
Die Kosten sind für eine Regelentwicklungszeit von vier Monaten mit fünf Mitarbeitern vorgesehen.
Für die individuellen Kosten wurden exemplarische Anbieter ausgewählt.

\begin{table} [H]
    \centering
    \begin{tabularx}{\textwidth}{|P{1.5cm}|m{2.5cm}|X|P{1.5cm}|P{2cm}|}
        \hline
        \textbf{Position} & \textbf{Art} & \textbf{Beschreibung} & \textbf{Menge} & \textbf{Kosten} \\
        \hline
        1 & Arbeitszeit &
        \begin{itemize}
            \item 5 Mitarbeiter
            \item 7 Arbeitsstunden pro Tag
            \item 73 Arbeitstage
            \item Stundenlohn von 20€
        \end{itemize}
        & 2.555 & 51.100 € \\
        \hline
        2 & Mietkosten &
        Unsere Bürofläche beträgt 60 m$^2$, der durchschnittliche Mietpreis in Bremen beträgt in der Überseestadt
        beträgt 13.40 € / m$^2$.
        & 4 & 3.216 € \\
        \hline
        3 & Stromkosten &
        Der durchschnittliche Stromverbrauch von Büroflächen beträgt 55 kWh / m$^2$ pro Jahr.
        Bei dem durchschnittlichen Strompreis von 80 € pro Monat für den Verbrauch.
        & 4 & 320 € \\
        \hline
        4 & Heizungskosten &
        Für die Fläche von 60m$^2$ fallen Heizungskosten von 55 € monatlich an.
        & 4 & 220 € \\
        \hline
        5 & Sonstige \newline Nebenkosten &
        Neben den Strom- und Heizungskosten fallen weiter Kosten von 90 € monatlich an.
        & 4 & 360 € \\
        \hline
        6 & Lizenzgebühren &
        Für die Entwicklung benötigt jeder Entwickler ein Abonnement für die Entwicklungsumgebung IntelliJ Idea.
        Die Kosten hierfür belaufen sich auf 352 € pro Jahr.
        Für die vier Monate sind die Kosten somit 117,33 €.
        & 5 & 586,65 € \\
        \hline
        7 & GPS-Tracker &
        Ein GPS-Tracker für jedes geplante Fahrzeug bei einem Einkaufspreis von 19,59 €
        & 250 & 4.897,5 € \\
        \hline
        8 & RFID-Karte &
        Eine RFID-Karte für jedes geplantes Mitglied bei einem Einkaufspreis von 0,80 €
        & 50.000 & 40.000 € \\
        \hline
        9 & RFID-Lesegerät &
        Ein RFID-Lesegerät für jedes geplante Fahrzeug bei einem Einkaufspreis 19,90 €
        & 250 & 4.975 € \\
        \hline
        10 & Risikokosten &
        Kosten aus der Risikoanalyse in \autoref{sec:risiko_analyse}
        & 1 & ? \\ %TODO Risikokosten
        \hline
        \multicolumn{4}{|r|}{\textbf{Gesamtkosten}} & 3 € \\
        \hline
    \end{tabularx}
    \caption{Kosten}
    \label{tab:Kosten}
\end{table}