Um das Projekt erfolgreich umzusetzen, ist es wichtig, potenzielle Risiken zu identifizieren und angemessene Maßnahmen zu ergreifen, um ihnen proaktiv entgegenwirken zu können.
Daher wurde eine umfassende Analyse durchgeführt, um die möglichen Risiken bei der Entwicklung einer Carsharing-Anwendung zu ermitteln.
Im Folgenden werden die fünf größten Risiken des Projekts vorgestellt:

\subsection*{1) Einsatz von Flutter}
Wir haben uns für den Einsatz von Flutter zur Entwicklung der Benutzeroberfläche entschieden, da es ein modernes Framework ist, das plattformübergreifende Programmierung ermöglicht.
Durch den Einsatz von Flutter können wir die Entwicklungszeit und -kosten optimieren, da wir die gleiche Codebasis für verschiedene Plattformen nutzen können.
Dies ermöglicht es uns, dem Kunden ohne zusätzlichen Aufwand sowohl eine Webanwendung als auch eine App zur Verfügung zu stellen.
Allerdings müssen wir berücksichtigen, dass nicht alle Entwickler in unserem Team bereits mit Flutter vertraut sind.
Dadurch entsteht das Risiko, dass diese möglicherweise Schwierigkeiten haben, die Sprache schnell und effizient zu erlernen, was potenziell zu Verzögerungen im Projekt führen könnte.
Zusätzlich besteht die Möglichkeit, dass nicht alle bewährten Standards angewendet oder effektive Lösungen gefunden werden, was sich wiederum auf die Qualität des Codes bzw.\ der Anwendung auswirken könnte.
Um diese Risiken zu minimieren, werden wir folgende Gegenmaßnahmen ergreifen:
\begin{enumerate}
    \item \textbf{Einsteiger Crashkurs}: Wir werden einen Einsteiger Crashkurs anbieten, um Entwicklern, die noch nicht mit Flutter vertraut sind, einen schnellen Einstieg in die Technologie zu ermöglichen. Der Crashkurs umfasst Schulungen, Tutorials und praktische Übungen, die darauf abzielen, die Grundlagen von Flutter zu vermitteln und den Entwicklern die erforderlichen Kenntnisse und Fähigkeiten zu vermitteln.
    \item \textbf{Wissenstransfer}: Erfahrene Entwickler werden ihr Fachwissen aktiv mit anderen Teammitgliedern teilen, um das Wissen über Flutter innerhalb des Teams zu erweitern. Dieser Wissenstransfer kann durch interne Schulungen, regelmäßige Wissensaustauschtreffen und Mentoring-Programme erfolgen. Dadurch wird sichergestellt, dass das Team schnell und effektiv die erforderlichen Kenntnisse entwickelt.
    \item \textbf{Entwicklung von Best Practices und Coding-Standards}: Wir werden gemeinsam interne Best Practices und Coding-Standards für die Verwendung von Flutter entwickeln. Hierdurch stellen wir sicher, dass bewährte Standards angewendet werden und die Codequalität von Anfang an auf einem hohen Niveau ist.
    \item \textbf{Regelmäßige Code-Reviews}: Wir werden regelmäßige Code-Reviews fest in unseren Entwicklungsprozess integrieren. Erfahrene Entwickler werden den Code systematisch auf Lesbarkeit, Effizienz, Sicherheit und die Einhaltung unserer Coding-Standards überprüfen. Durch konstruktives Feedback und mögliche Verbesserungs- oder Optimierungsvorschläge unterstützen sie Entwickler in ihrer Lernphase. Unsere regelmäßigen Code-Reviews haben außerdem das Ziel, die Codequalität kontinuierlich zu verbessern, indem potenzielle Fehler frühzeitig erkannt und behoben werden. Durch diesen Prozess haben die Entwickler zudem die Möglichkeit, ihr Wissen und ihre Fähigkeiten zu erweitern, indem sie von den Erfahrungen und dem Fachwissen ihrer Kollegen profitieren.
\end{enumerate}

\subsection*{2) Personelle Engpässe}
Ein weiteres bedeutsames Projektrisiko sind personelle Engpässe.
Wenn das Team nicht über ausreichend Ressourcen verfügt, um die erforderlichen Funktionen und Komponenten der Anwendung zu entwickeln, kann es zu Verzögerungen im Zeitplan kommen.
Dies kann wiederum das Gesamtergebnis des Projekts beeinträchtigen und dazu führen, dass der Kunde unzufrieden ist.
Ein weiterer Aspekt ist die Qualität der entwickelten Anwendung.
Personelle Engpässe können zudem dazu führen, dass das Team überlastet ist.
Infolgedessen besteht das Risiko, dass den Entwicklern nicht ausreichend Zeit zur Verfügung steht, um umfangreiche Tests durchzuführen.Dies wiederum kann dazu führen, dass potenzielle Fehler oder Sicherheitslücken nicht rechtzeitig identifiziert werden.
Es besteht die Gefahr, dass die Anwendung nicht den Anforderungen des Kunden entspricht oder möglicherweise Sicherheitsrisiken aufweist.
Wir werden folgende Gegenmaßnahmen ergreifen:
\begin{enumerate}
    \item \textbf{Vermeidung von Schlüsselpersonen}: Wir streben an, nicht ausschließlich von einer einzigen Schlüsselperson abhängig zu sein. Durch eine breitere Verteilung des Fachwissens und der Verantwortlichkeiten im Team stellen wir sicher, dass das Projekt auch bei Abwesenheit oder Ausfall einzelner Teammitglieder kontinuierlich voranschreiten kann.
    \item \textbf{Priorisierung und Fokussierung}: Eine klare Priorisierung der Aufgaben und Konzentration auf die wesentlichen Funktionen und Komponenten der Anwendung, um eine effiziente Nutzung der verfügbaren Ressourcen zu gewährleisten.
    \item \textbf{Flexibles Projektmanagement}: Das Projektmanagement sollte flexibel sein und in der Lage sein, auf unvorhergesehene Ereignisse und Änderungen zu reagieren. Eine kontinuierliche Überwachung des Projektfortschritts ermöglicht die frühzeitige Erkennung von Engpässen und die rechtzeitige Anpassung der Ressourcenverteilung.
    \item \textbf{Externe Unterstützung}: Bei anhaltenden personellen Engpässen kann die Zusammenarbeit mit externen Dienstleistern oder das Outsourcing bestimmter Aufgaben in Betracht gezogen werden. Dies ermöglicht den Zugang zu zusätzlicher Expertise und Ressourcen.
\end{enumerate}

\subsection*{3) Externe Abhängigkeiten durch Dienstleister}
Da wir in spezialisierten und komplexen Bereichen wie der Zahlungsabwicklung und Authentifizierung über begrenzte Kenntnisse verfügen, haben wir uns entschieden, diese Verantwortlichkeiten an externe Dienstleister zu übertragen.
Diese Fachleute zeichnen sich nicht nur durch technische Expertise und ein tiefes Verständnis der rechtlichen Vorschriften in ihrem Gebiet aus, sondern sind auch für die Einhaltung höchster Sicherheitsanforderungen verantwortlich.
Obwohl die Zusammenarbeit mit externen Experten viele Vorteile bietet, ist es wichtig, auch die damit verbundenen Risiken zu berücksichtigen.
Eines der Hauptprobleme ist die mögliche Abhängigkeit vom Dienstleister.
Sollte dieser ausfallen oder seine Leistungen nicht mehr bereitstellen, hätte dies erhebliche Auswirkungen auf die Carsharing-Anwendung.
Die Kontrolle über die Dienstleister ist ein weiteres Problem.
Da sie nicht direkt unserer Unternehmensführung unterliegen, können wir ihre Arbeitsweise und Prozesse nicht vollständig kontrollieren.
Dies kann zu Unstimmigkeiten in der Leistungserbringung und Qualitätsstandards führen.
Um den genannten Risiken in Bezug auf die Abhängigkeit von externen Dienstleistern und die fehlende Kontrolle über ihre Arbeitsweise entgegenzuwirken, können verschiedene Gegenmaßnahmen ergriffen werden.
Hier sind einige mögliche Ansätze:
\begin{enumerate}
    \item \textbf{Sorgfältige Auswahl der Dienstleister}: Eine gründliche Due-Diligence-Prüfung der potenziellen Dienstleister, einschließlich Überprüfung ihrer technischen Fähigkeiten, Erfahrung, Sicherheitsmaßnahmen, Zertifizierungen und Referenzen, gewährleistet die Auswahl vertrauenswürdiger Partner, die unseren Anforderungen gerecht werden.
    \item \textbf{Regelmäßige Kommunikation und Zusammenarbeit}: Durch eine enge Zusammenarbeit und regelmäßige Kommunikation mit den Dienstleistern wird eine effektive Kontrolle über ihre Arbeitsweise und Prozesse sichergestellt. Regelmäßige Besprechungen, Berichterstattung und Austausch helfen dabei, Missverständnisse zu vermeiden und die Leistungserbringung sowie die Einhaltung von Qualitätsstandards zu überwachen.
\end{enumerate}

\subsection*{4) Hardwarelieferanten liefern zu spät oder mit unzureichender Qualität}
Für die Lieferung der Hardware werden wir ebenfalls auf externe Unternehmen zurückgreifen.
Wie bereits erwähnt, birgt die Zusammenarbeit mit einem Dritten Risiken.
In Bezug auf den GPS-Tracker-Lieferanten besteht das Risiko, dass die gelieferten Tracker eine unzureichende Qualität aufweisen könnten.
Ein zusätzliches Risiko ergibt sich aus möglichen Verzögerungen bei der Lieferung der Tracker.
Solche Verzögerungen könnten unseren Zeitplan für das Projekt beeinträchtigen.
Folgende Maßnahmen ergreifen wir, um diese Risiken zu minimieren:
\begin{enumerate}
    \item \textbf{Sorgfältige Auswahl des Lieferanten}: Wir wählen den Lieferanten sorgfältig aus, indem wir seine Zuverlässigkeit, Vertrauenswürdigkeit und Qualitätsbewusstsein bewerten.
    \item \textbf{Festlegung klarer Qualitätsstandards}: Wir definieren detaillierte Qualitätsanforderungen und Spezifikationen für die gelieferten Tracker. Dadurch stellen wir sicher, dass die Tracker unseren Anforderungen entsprechen und eine zuverlässige Leistung bieten.
    \item \textbf{Verträge und Service Level Agreements (SLAs)}: Wir schließen vertragliche Vereinbarungen mit dem Lieferanten, die klare Qualitätsstandards, Lieferfristen, Eskalationsverfahren und mögliche Sanktionen bei Nichterfüllung enthalten.
\end{enumerate}

\subsection*{5) Kurzfristige Anforderungsänderungen}
Kurzfristige Anforderungsänderungen seitens des Kunden stellen ein Risiko dar, da sie Auswirkungen auf den Zeitplan, das Budget und die Ressourcen haben können.
Um diese Risiken zu minimieren, ergreifen wir folgende Maßnahmen:
\begin{enumerate}
    \item \textbf{Klare Kommunikation und Anforderungsmanagement}: Durch regelmäßige Meetings und einen offenen Austausch mit dem Kunden stellen wir sicher, dass Anforderungen klar definiert und dokumentiert werden. Dies ermöglicht es uns, Änderungen frühzeitig zu erkennen und angemessen darauf zu reagieren.
    \item \textbf{Agile Projektmethodik}: Indem wir eine agile Vorgehensweise wie beispielsweise Scrum verwenden, können wir flexibel auf kurzfristige Anforderungsänderungen reagieren. Durch die Aufteilung des Projekts in iterative Sprints und regelmäßige Rückkopplungsschleifen können wir Anpassungen zeitnah umsetzen.
    \item \textbf{Priorisierung der Anforderungen}: Bei kurzfristigen Anforderungsänderungen bewerten wir deren Auswirkungen auf den Zeitplan und das Budget. Durch eine klare Priorisierung können wir sicherstellen, dass die wichtigsten Änderungen priorisiert und umgesetzt werden, während weniger kritische Änderungen möglicherweise auf spätere Phasen des Projekts verschoben werden.
\end{enumerate}