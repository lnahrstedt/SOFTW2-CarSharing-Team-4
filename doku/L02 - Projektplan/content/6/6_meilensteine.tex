Für die Planung des Projekts werden Meilensteine genutzt, um es in klar definierte Abschnitte zu unterteilen.
Dadurch vereinfacht sich die Organisation sowie die Überwachung des Projekts.
Die einzelnen Meilensteine wurden als messbare Ziele definiert,
um eine objektive Bewertung des Projektfortschritts zu ermöglichen.
Die im Voraus gesetzten Termine für die Vollendung eines Meilensteins dienen dem Zeitmanagement und helfen somit
bei der Einteilung von Ressourcen für die Meilensteine.

\begin{table} [H]
    \centering
    \begin{tabularx}{\textwidth}{|m{3.5cm}|X|P{2cm}|}
        \hline
        Meilenstein & Beschreibung & Plantermin \\
        \hline
        Beginn der \newline Entwicklung & Die tatsächliche Umsetzung des Projekts beginnt, nachdem das Pflichtenheft
        und der Projektplan erarbeitet wurden & 26.05.23 \\
        \hline
        Mindestanforderungen & Die im Pflichtenheft definierten Anforderungen mit der Priorität
        \enquote{Muss} sind vollständig implementiert und getestet.\ & 20.06.23 \\
        \hline
        Fertigstellung des \newline Backends & Alle im Pflichtenheft definierten Anforderungen,
        die das Backend betreffen, wurden implementiert und getestet.\ & 07.07.23 \\
        \hline
        Fertigstellung des \newline Frontends & Alle im Pflichtenheft definierten Anforderungen,
        die das Frontend betreffen, wurden implementiert und getestet.\ & 07.07.23 \\
        \hline
        Präsentation & Das Projekt wurde dem Kunden vorgestellt.\ Die Implementierung wurde vorgeführt.\   & 13.07.23 \\
        \hline
        Technische \newline Dokumentation & Die Dokumentation, welches unter anderem die Architekturbeschreibung und
        Verifikation beinhaltet, wurde dem Kunden ausgeliefert.\ & 20.07.23 \\
        \hline
        Implementierung, \newline Test \& Abnahme & Das Projekt wurde auf ein Git-Repository übertragen,
        eine einweisende README-Datei wurde verfasst, die Installation sowie Implementierung wurde dem Kunden vorgeführt.\ & 20.07.23 \\
        \hline
    \end{tabularx}
    \caption{Meilensteine}
    \label{tab:meilensteine}
\end{table}
