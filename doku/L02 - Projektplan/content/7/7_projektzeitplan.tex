Zur besseren Planung und Visualisierung des Projekts wurde ein Gantt-Diagramm entwickelt.
Dieses Diagramm ist unter \autoref{uml:zeitplan} zu finden.
Bei der Modellierung wurde die technische Dokumentation aufgrund der ständigen Aktualisierung während
der Implementierung als eine lange Aufgabe aufgefasst.
Diese ist in Grün dargestellt.
Dazu gehört das Product Backlog, welches automatisch anhand der geplanten Aufgaben erstellt wird,
die Beschreibung der ausgewählten Technologien und Werkzeuge, der Git-Repository-Struktur,
der Komponenten und Schnittstellen.
Die restlichen Teile der technischen Dokumentation wurden separat im Gantt-Diagramm aufgefasst und haben dieselbe Farbe
wie die Überaufgabe. \medskip

Das Projekt startet mit dem Meilenstein \enquote{Beginn der Entwicklung} , um den Start zu markieren.
Daraufhin folgt das Modellieren, was zum Beispiel die Erstellung von Klassen- und Sequenzdiagrammen beinhaltet.
Infolgedessen startet die Entwicklung der Mindestanforderungen.
Auch hier sind alle zugehörigen Aufgaben mit der Farbe Orange gekennzeichnet. \medskip

Nach der Implementierung der Mindestanforderungen wurde der Meilenstein \enquote{Mindestanforderungen} erreicht.
Danach folgt das Fertigstellen des Backends und des Frontends.
Die Aufgaben zum Backend sind dabei alle in Blau gefärbt und die des Frontends in Gelb. \medskip

Nach Abschluss dieser Aufgaben folgen drei Meilensteine.
Dabei handelt es sich um \enquote{Fertigstellung des Backends} , \enquote{Fertigstellung des Frontends} und \enquote{Präsentation}.
Bei der Präsentation handelt es sich um die Vorstellung des gesamten Projekts und der Implementierung. \medskip

Daraufhin folgt das Verifizieren, als ein Schritt der technischen Dokumentation.
Dies entspricht der Beschreibung der Tests und der Testergebnisse und der Erstellung einer Verification Matrix. \medskip

Zum Abschluss der technischen Dokumentation werden die Meilensteine \enquote{Technische Dokumentation} und
\enquote{Implementierung, Test \& Abnahme} erreicht.
Diese stellen gleichzeitig den Abschluss des Projekts dar.