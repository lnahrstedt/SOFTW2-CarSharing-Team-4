Die im Folgenden aufgelisteten Technologien und Werkzeuge, inklusive Beschreibung, finden in dem Fastlane-Projekt Anwendung. \\

\textbf{IntelliJIDEA-IDE} ist eine integrierte Entwicklungsumgebung (IDE), die speziell für die Java-Entwicklung konzipiert ist.
Es ist eine leistungsstarke und weit verbreitete IDE, die Funktionen und Tools bietet, um den Entwicklungsprozess effizienter und produktiver zu gestalten.
IntelliJIDEA wurde von uns sowohl für das Backend, als auch das Frontend verwendet.\\

\textbf{Android Studio} ist eine integrierte Entwicklungsumgebung, die speziell für die Entwicklung von Android-Apps entwickelt wurde.
Es basiert auf der IntelliJ IDEA-Plattform und bietet umfangreiche Funktionen und Tools, um den gesamten Entwicklungsprozess für Android-Anwendungen zu unterstützen.
Da wir Fastlane Plattformunabhängig entwickeln, soll dieses auch auf Android-Tablets und Smartphones laufen.
Diesbezüglich fungiert Android Studio als Entwicklungsumgebung mit den notwendigen Tools.\\

\textbf{Visual Studio Code} ist ein plattformübergreifender, quelloffener Code-Editor.
Es ist eine leichte IDE mit etwaigen Funktionen, die die Entwicklung und Bearbeitung von Code in verschiedenen Programmiersprachen unterstützen.
Konkret wurde Visual Studio Code mit der \enquote{REST CLient} Erweiterung verwendet, um die REST-Schnittstellen des Backends zu testen.\\

\textbf{Java} ist eine objektorientierte Programmiersprache und findet Anwendung im Backend.
Sie zeichnet sich unter anderem durch Plattformunabhängigkeit, Zuverlässigkeit,
Robustheit und Sicherheit aus.
Zudem ist Java eine der meistverbreiteten Programmiersprachen der Welt und damit zukunftssicher, weshalb sich diese für dieses Projekt anbietet.\\

\textbf{Apache Maven} ist ein in der Programmiersprache Java geschriebenes Kommandozeilenwerkzeug aus der Kategorie
der Build-Werkzeuge.
Maven erleichtert uns durch Dependencies das Einbinden von Bibliotheken.
Die Dependencies werden in der pom.xml-Datei des Projektes aufgeführt.\\

\textbf{Spring} ist ein umfangreiches Java-Framework für die Entwicklung von Enterprise-Anwendungen.
Unter Enterprise-Anwendungen versteht man komplexe Softwarelösungen, welche der Automatisierung und Integration von Geschäftsprozessen und Datenverwaltung dienen.
Konkret in diesem Projekt wird Spring verwendet, um die REST-API bereit zustellen.
Darüber hinaus bietet das Spring-Framework etwaige Module und Erweiterungen, die für spezifische Anwendungsfälle wie Spring Security und Spring Boot entwickelt wurden.\\

\textbf{Spring Boot} ist ein Framework, das auf dem Spring-Framework aufbaut.
Durch diese Lösung wird der Aufwand, der von uns für die Konfiguration und Bereitstellung von Spring-Anwendungen aufgebracht werden muss, deutlich reduziert.\\

\textbf{Spring Security} ist ein leistungsstarkes Sicherheitsframework für Java-basierte Anwendungen, das entwickelt wurde,
um die Implementierung von Authentifizierung, Autorisierung und anderen Sicherheitsfunktionen zu vereinfachen.
Es integriert sich nahtlos mit dem Spring-Framework und bietet umfassende Funktionen zur Sicherung von Anwendungen.\\

\textbf{JSON} (JavaScript Object Notation) ist ein leichtgewichtiges Datenformat, das zur strukturierten Darstellung von Daten verwendet wird.
Es basiert auf der Syntax von JavaScript, ist jedoch unabhängig von einer bestimmten Programmiersprache.
In diesem Projekt wird ein \enquote{REST (Representational State Transfer)-API (Application Programming Interface)} verwendet,
wobei das Frontend mit dem Backend über REST-Schnittstellen kommuniziert.
JSON wird als Zwischenformat verwendet, um unter anderem Nutzerdaten zwischen den beiden Systemen, bzw. Programmiersprachen zu konvertieren.\\

\textbf{fasterxml/Jackson} ist eine leistungsstarke und weit verbreitete Java-Bibliothek zur Verarbeitung von JSON-Daten.
Es ermöglicht das Lesen, Schreiben und Manipulieren von JSON in Java-Objekte und umgekehrt.\\

\textbf{Dart} ist eine Programmiersprache, zur Entwicklung von plattformübergreifenden Anwendungen.
Sie ist die Basis für das Flutter-Framework.\\

\textbf{Flutter} ist ein Open-Source-Framework, das zur Erstellung plattformübergreifender, mobiler Anwendungen entwickelt wurde.
Mit Flutter ist es möglich native Benutzeroberflächen für iOS, Android, Windows und Web-Anwendungen aus einer einzigen Codebasis zu erstellen.
Mit Flutter wurde von uns das Frontend aufgesetzt. \\

\textbf{SQLite} wird von der Anwendung als eingebettete Datenbank verwendet.
Sie zeichnet sich durch Einfachheit, Portabilität und Geschwindigkeit aus.\\

\textbf{JUnit} ist ein Open-Source-Testframework, das von uns für Annotationen verwendet wurde.\\

\textbf{Lombok} ist eine Java-Bibliothek, die die Entwicklung der Java-Anwendung erleichtert, indem sie den Code für wiederkehrende Aufgaben reduziert.
Sie bietet Annotationen, die zur Generierung von Standardcode verwendet werden können. \\

\textbf{RESTAssured} haben wir für unsere Integrationstests genutzt.
Es ermöglicht uns, HTTP-Anfragen zu Testzwecken an die API zu senden und die entsprechenden Antworten zu überprüfen.\\

\textbf{nginx} ist ein leistungsstarker, Open-Source-Webserver und Reverse-Proxy-Server, der auch als Load Balancer, HTTP-Cache und Mail-Proxy eingesetzt werden kann.\\

\textbf{Confluence} ist eine kollaborative Plattform für das Team- und Projektmanagement.
Es ist eine Wiki-Software, die es dem Team und dem Kunden ermöglicht, gemeinsam an Inhalten zu arbeiten, Informationen zu teilen und Wissen rund um Fastlane zu dokumentieren.\\

\textbf{Jira} ist eine Projektmanagement-Software zur Verwaltung von Aufgaben, Projekten und Arbeitsabläufen.
Die Anforderungen, die wir bei Confluence gemeinsam mit dem Kunden festgelegt haben, wurden in Jira als Tasks überführt und konnten so unabhängig und systematisch von Teammitgliedern abgearbeitet werden.\\

\newpage

\textbf{Bitbucket} ist eine webbasierte Plattform für die Versionsverwaltung von Quellcode und die Zusammenarbeit an dem Softwareprojekt.
Es bietet Funktionen für das Hosting vom Git-Repository, die Verfolgung von Änderungen, Pull-Requests, Code-Reviews und die Integration mit anderen Entwicklungstools.
Bearbeitet ein Teammitglied einen Task von Jira, so wurde dafür ein Branch auf Bitbucket erstellt in dem unabhängig vom restlichen Projekt gearbeitet werden konnte.
Dies vermeidet größere Fehler, durch beispielsweise Merge-Konflikte auf dem Master.\\

\textbf{Git-Repository} ist ein Speicherort, in dem Git die Versionshistorie und den Quellcode eines Projekts speichert.
Es enthält alle Dateien der Fastlane-Anwendung und Ordner und Informationen, die für die Versionsverwaltung mit Git benötigt werden.\\

\textbf{Docker} ist eine Open-Source-Plattform, um Anwendungen in Containern zu isolieren und bereitzustellen.
Mithilfe von Docker können wir die Anwendung und ihre Abhängigkeiten in standardisierte Container verpacken, die dadurch unabhängig von der zugrunde liegenden Infrastruktur lauffähig ist.\\

\textbf{Docker Compose}  ist ein Tool, das die Verwaltung und Orchestrierung von mehreren Docker-Containern ermöglicht.
Mit Docker Compose können wir eine Anwendung in einer einzigen Konfigurationsdatei beschreiben, die als \enquote{docker-compose.yml} zu finden ist.
Diese Datei definiert die Container, ihre Konfigurationen, Netzwerke und Abhängigkeiten.\\

\textbf{Docker Desktop} ist eine Softwareanwendung, die die Verwendung von Docker auf Desktop-Computern ermöglicht.
Es handelt sich um eine Benutzeroberfläche und ein Toolset, das Docker-Container auf lokalen Maschinen einrichten, verwalten und ausführen kann.
In Docker Desktop sind alle vorhandenen Container einer Maschine aufgelistet, was uns während der Entwicklung eine bessere Übersicht und Verwaltung aller Container ermöglicht.\\

\textbf{Figma} ist ein browserbasiertes Design- und Prototyping-Tool.
Es ermöglicht das zusammenzuarbeiten in Echtzeit, Designs zu erstellen, zu bearbeiten und zu teilen.
Über Figma wurden erste Designideen für das Frontend entwickelt und festgehalten.
Sobald wir mit einem Design zufrieden waren, wurde dieses beispielsweise durch Widgets im Frontend umgesetzt.\\

\textbf{Materialdesign} ist ein Designkonzept für die Gestaltung von Benutzeroberflächen in Apps und Websites.
Es bietet eine kohärente Designsprache und eine umfassende Designrichtlinie für die Entwicklung von Apps und Websites.
Materialdesign wurde von uns, neben Figma für das Design des Frontends verwendet.
Jedoch wurde weitestgehend mit Figma gearbeitet. \\

\textbf{Visual Paradigm} ist eine umfassende Software-Suite für das Modellieren und die Entwicklung von Softwarelösungen.
In Zusammenhang mit dem Fastlane-Projekt stellt dieses Tools und Funktionen für die UML-Modellierung zur Verfügung.\\

\newpage