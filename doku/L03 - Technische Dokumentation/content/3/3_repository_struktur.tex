Das Entwicklerteam nutzt Atlassian-Produkte, darunter Confluence und Jira, und hat sich daher für Bitbucket als Versionsverwaltung entschieden.
Die Entscheidung basiert auf den Vorteilen, dass erstellte Aufgaben über Jira direkt als Branches in Bitbucket erstellt werden können.
Dies optimiert die Projektverwaltung und Planung, was zu Zeitersparnis und gesteigerter Effizienz führt.
Bitbucket fungiert ausschließlich als Hosting-Plattform für Git-Repositories.
Die Entwickler arbeiten dabei mit sogenannten Branches im Git-Repository.
Wurde ein Branch auf Bitbucket erstellt, so konnte von diesem lokal eine Kopie angelegt werden, um Änderungen zu speichern und anschließend in das Git-Repository hochzuladen.
Für jeden erstellten Task in Jira wurde auf Bitbucket ein eigener Feature-Branch erstellt.
Dadurch wird ermöglicht, dass die Aufgaben unabhängig voneinander bearbeitet werden können und das Risiko schwerwiegender Fehler, beispielsweise durch Merge-Konflikte auf dem Master-Branch, verringert wird.

Um einen Merge auf den Master-Branch durchzuführen, war es erforderlich, einen Merge-Request im jeweiligen Feature-Branch zu erstellen.
Auf dem Master-Branch wurden ausschließlich funktionierende Features gepusht und zusammengeführt.
Dadurch wurde sichergestellt, dass nur vollständig getestete und funktionsfähige Änderungen im Master-Branch vorhanden sind.\\

Das Repository ist in den Sektionen \textbf{code} und \textbf{doku} unterteilt.
Diese Aufteilung ermöglicht eine strukturierte Verwaltung der Projektressourcen und eine klare Trennung zwischen der Dokumentation und dem eigentlichen Code.
Dadurch wird die Zusammenarbeit und Entwicklung erleichtert.
In der Sektion \enquote{doku} befinden sich die bearbeiteten Laboraufgaben.
Die Sektion \enquote{code} ist unterteilt in frontend und backend.  \\
Für das \enquote{Frontend} wurde \enquote{Flutter} verwendet, diesbezüglich gibt es Unterordner für verschiedene Plattformen, wie Ios, Android und windows.
In dem Ordner \enquote{lib} befinden sich alle relevanten Dateien und Ordner für die Entwicklung der Benutzeroberfläche.
Die Aufteilung erfolgt in die Ordner \enquote{Constants}, \enquote{control}, \enquote{data}, \enquote{model}, \enquote{widgets} und der Main-Datei \enquote{mainDashboard}.    \\
Die Sektion \enquote{bakend} enthält einen Ordner \enquote{Entwurf} in dem alle Entwürfe hinterlegt sind und den Ordner \enquote{fastlane}
in dem alle Dateien und Ordner für die Entwicklung der serverseitigen Logik vorhanden sind.
Die wichtigsten Daten in diesem Ornder sind die Datenbank, ein Dockerfile, ein Ordner namens \enquote{requests} zum Testen der Schnittstellen und der Source-Ordner \enquote{src}.
Im Source-Ornder befindet sich neben dem Main-Ornder ein Test-Ordner, in dem unter anderem alle Kontroller getestet werden.
Der Main-Ordner enthält die ausführbare Datei \enquote{FastlaneApplication} sowie Ordner, die alle Funktionen und Daten für das Backend beinhalten.




