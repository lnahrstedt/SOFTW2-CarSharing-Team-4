\textbf{Was testen wir?}\smallskip

Um sicherzustellen, dass alle vom Backend bereitgestellten Schnittstellen ordnungsgemäß funktionieren,
haben wir uns dazu entschlossen, alle Schnittstellen zu testen.
Zudem haben wir festgelegt, dass wir im Frontend die Mindestanforderungen testen.

\bigskip
\textbf{Welche Teststrategie wenden wir an?}\smallskip

Für die Backendtests haben wir uns für den ausschließlichen Einsatz von automatisierten Integrationstests entschieden.
Der Hauptgrund für diese Entscheidung ist, dass diese eine ausgewogene Balance zwischen Codeabdeckung und der Prüfung der korrekten Kommunikation zwischen den verschiedenen Schichten bieten.
Für komplexere Methoden (zum Beispiel einige Update-Methoden), die typischerweise durch Unit-Tests abgedeckt werden sollten, haben wir uns für einen Kompromiss entschieden.
Solche Methoden werden durch eine hinreichende Anzahl von Integrationstests, die auch Randfälle berücksichtigen, auf mögliche Fehlfunktionen hin untersucht.\medskip

Aufgrund zeitlicher Begrenzungen haben wir uns dazu entschlossen, die Funktionalität des Frontends ausschließlich durch manuelle Tests zu prüfen.

\bigskip
\textbf{Wie testen wir? Was ist unser Workflow?}\smallskip

Nach der Implementierung einer Schnittstelle haben wir diese unmittelbar getestet.
Hierfür haben wir Testszenarien nach der \enquote{Given-When-Then}-Methode entwickelt
und anschließend jedes Szenario in einzelne Tests aufgeteilt.
Bei jeder neuen Version wurden alle Tests ausgeführt.
Falls ein Test fehlschlug, konnte der Fehler aufgrund der atomaren Natur schnell lokalisiert und behoben werden. \medskip

Für die Testung des Frontends haben wir uns ebenfalls Szenarien nach der \enquote{Given-When-Then}-Methodik ausgedacht und diese im Nachhinein manuell durchgetestet.

\bigskip
\textbf{Wie dokumentieren wir?}\smallskip

Wie zuvor angeführt, ist es die Aufgabe jedes Entwicklers, nach der Umsetzung einer Schnittstelle Testfälle dafür zu erstellen.
Wir haben uns darauf verständigt, diese Testfälle sorgfältig zu protokollieren, um deren Nachvollziehbarkeit zu gewährleisten.
Dabei galt es, das folgende Schema zu befolgen: \bigskip

\resizebox{\textwidth}{!}{%
    \begin{tabular}{|l|cl|cl|cl|l|}
        \hline
        &
        \multicolumn{2}{c|}{Given} &
        \multicolumn{2}{c|}{When} &
        \multicolumn{2}{c|}{Then} &
        \\ \hline
        \multicolumn{1}{|c|}{Testnummer} &
        \multicolumn{1}{c|}{Beschreibung} &
        \multicolumn{1}{c|}{Rolle} &
        \multicolumn{1}{c|}{Request Method} &
        \multicolumn{1}{c|}{Endpunkt} &
        \multicolumn{1}{c|}{erwarteter   Statuscode} &
        \multicolumn{1}{c|}{erwartete   Ausgabe im Body} &
        \multicolumn{1}{c|}{OK?} \\ \hline
        &
        \multicolumn{1}{l|}{} &
        &
        \multicolumn{1}{l|}{} &
        &
        \multicolumn{1}{l|}{} &
        &
        \\ \hline
    \end{tabular}%
}

\bigskip
\textbf{Welche Werkzeuge nutzten wir zur Durchführung der Tests?}\smallskip

Für die Durchführung unserer automatisierten Tests haben wir eine Kombination aus verschiedenen Bibliotheken verwendet.
Einige dieser Bibliotheken sind bereits integraler Bestandteil von Spring Boot.
Spring Boot bietet außerdem die Möglichkeit, zu Testzwecken eine reduzierte Version der Anwendung unter einem anderen Port zu starten.
Zur Steuerung der Testreihenfolge haben wir Annotationen aus der \enquote{JUnit 5} Bibliothek verwendet.
Die Integrationstests wurden mithilfe der Bibliothek \enquote{RESTAssured} durchgeführt, welche alle erforderlichen Methoden zur Simulation von API-Anfragen nach der Methodik \enquote{Given-When-Then} bereitstellt.
Schließlich verwendeten wir Hamcrest, um die tatsächlichen Rückgabewerte mit den erwarteten Ergebnissen abzugleichen.

\bigskip
\textbf{Benötigen wir Testdaten? Wenn ja, wie generieren wir solche?}\smallskip

Einige unserer Tests, wie zum Beispiel das Löschen eines Accounts, setzen voraus, dass bereits Accounts in der Datenbank vorhanden sind.
Daher war es erforderlich, das Vorhandensein solcher Daten zu simulieren.
Hierfür haben wir den Dienst \enquote{Mockaroo} in Anspruch genommen.
\enquote{Mockaroo} ist ein Online-Tool, das entwickelt wurde, um realistische Testdaten in verschiedenen Formaten (wie CSV, JSON, SQL und Excel) zu generieren.
Vor der Durchführung des ersten Tests haben wir all diese Daten in die Datenbank eingepflegt, um eine adäquate Testumgebung zu schaffen.