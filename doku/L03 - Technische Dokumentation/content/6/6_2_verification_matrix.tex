Im Folgenden finden sie ein umfassende Tabelle, die alle entweder im Frontend oder im
Backend getesteten Anforderungen enthält.
Des Weiteren wird in den Spalten \enquote{Testnummer Backend} und \enquote{Testnummer Frontend} die Nummer
des jeweiligen Tests erfasst.
Die Nummern für die Backend Tests sind in \hyperref[subsec:testfaelle_backend]{Testfälle Backend} wiederzufinden.
Die Nummern der Frontend Tests sind in \hyperref[subsec:testfaelle_frontend]{Testfälle Frontend} zu finden.

\begin{longtable}{|m{2cm}|m{6.5cm}|m{2.5cm}|m{2.5cm}|}
    \hline
    \textbf{ID} & \textbf{Beschreibung} & \textbf{Testnummer Backend} & \textbf{Testnummer Frontend} \\
    \hline
    FA-5 & Mitgliedskonten dürfen nur angelegt werden, wenn die Person mindestens 18 Jahre alt ist & 2 & 2 \\
    \hline
    FA-6 & Die Anwendung soll über eine URL erreichbar sein & Durch alle Tests gewährleistet & - \\
    \hline
    FA-29 & Mitgliedskonten müssen durch einen Mitglied angelegt werden können & 54 & 1 \\
    \hline
    FA-30 & Ein Admin muss Mitarbeiterzugänge anlegen können & 61 & - \\
    \hline
    FA-32 & Mitgliedsdaten (außer Kundennummer \& Identifikationsdaten) müssen durch einen Mitarbeiter geändert werden können & 16 & - \\
    \hline
    FA-33 & Mitgliedsdaten (außer Kundennummer \& Identifikationsdaten) müssen durch ein Mitglied geändert werden können & 17 & 15 \\
    \hline
    FA-34 & Ein Mitglied muss sich einloggen können & 28 & 3 \\
    \hline
    FA-35 & Ein Mitarbeiter muss sich einloggen können & 28 & 3 \\
    \hline
    FA-36 & Ein Admin muss sich einloggen können & 28 & 3 \\
    \hline
    FA-37 & Ein Mitglied muss sich ausloggen können & - & 5 \\
    \hline
    FA-38 & Ein Mitarbeiter muss sich ausloggen können & - & 5 \\
    \hline
    FA-39 & Ein Admin muss sich ausloggen können & - & 5 \\
    \hline
    FA-40 & Neue Buchungen müssen durch ein Mitglied erstellt werden können & 85 & 7 \\
    \hline
    FA-41 & Neue Buchungen müssen durch einen Mitarbeiter erstellt werden können & 87 & - \\
    \hline
    FA-42 & Buchungen müssen von einem Mitglied eingesehen werden können & 66 & 10, 12 \\
    \hline
    FA-43 & Buchungen müssen von einem Mitarbeiter eingesehen werden können & 65 & 10, 12 \\
    \hline
    FA-44 & Buchungen müssen von einem Mitglied storniert werden können & 104 & 11 \\
    \hline
    FA-45 & Buchungen müssen von einem Mitarbeiter storniert werden können & 105 & 11 \\
    \hline
    FA-46 & Ein Mitglied soll einen Tarif wählen können & 54,17 & 1, 15 \\
    \hline
    FA-50 & Ein Mitarbeiter soll Autos aus dem Fuhrpark bearbeiten können & 135 & - \\
    \hline
    FA-51 & Ein Mitarbeiter soll Autos aus dem Fuhrpark entfernen können & 139 & - \\
    \hline
    FA-57 & Buchungen müssen von einem Admin eingesehen werden können & 64 & 10, 12 \\
    \hline
    FA-64 & Vergangene Buchungen sollen durch ein Mitglied einsehbar sein & 66 & 10, 12 \\
    \hline
    FA-65 & Vergangene Buchungen sollen durch einen Mitarbeiter einsehbar sein & 65 & 10, 12 \\
    \hline
    FA-67 & Ein Mitarbeiter soll Autos dem Fuhrpark hinzufügen können & 130 & - \\
    \hline
    FA-68 & Buchungen müssen von einem Admin storniert werden können & 106 & 11 \\
    \hline
    FA-69 & Vergangene Buchungen sollen durch einen Admin einsehbar sein & 64 & 10, 12 \\
    \hline
\end{longtable}
\label{tab:verifikations_matrix}

Im Rahmen der Testphase wurden insgesamt 140 automatisierte Integrationstests und 16 manuelle Tests durchgeführt.
Die Verifikationsmatrix dokumentiert die Erfüllung der genannten Anforderungen, wobei einige Anforderungen ausschließlich im Backend und nicht im Frontend umgesetzt wurden.
Des Weiteren enthält die Verifaktionsmatrix Verweise auf die automatisierten Integrationstests des Backends sowie die manuellen Tests des Frontends.
Zusätzlich zu den in der Verifikationsmatrix aufgeführten Tests wurden weitere Testfälle durchgeführt.
Bei diesen Testfällen handelte es sich hauptsächlich um Grenzfall- und Sicherheitstests, die nicht eindeutig einer spezifischen Anforderung zugeordnet werden konnten.
Eine detaillierte Auflistung der Tests befindet sich in den Abschnitten \hyperref[subsec:testfaelle_backend]{Testfälle Backend}
und \hyperref[subsec:testfaelle_frontend]{Testfälle Frontend} im Anhang.
Alle durchgeführten Tests waren erfolgreich und belegen somit die Funktionsfähigkeit des Systems für die genannten Testfälle.