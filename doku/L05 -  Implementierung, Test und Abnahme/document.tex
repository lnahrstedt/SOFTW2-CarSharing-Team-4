\section{Auffinden der Anwendung}
\label{sec:auffinden}

Das Projekt mitsamt der Dokumentation ist in dem Git-Repository \enquote{SOFTW2 CarSharing Team 4} auf dem HSB Git-Server zu finden.
Von dort kann dieses geklont werden.

\section{Voraussetzungen}
\label{sec:voraussetzungen}

Für die Installation der Anwendung fastlane müssen folgende Voraussetzungen erfüllt sein, um eine problemfreie
Nutzung zu garantieren.

\subsection{Hardware}
\label{subsec:hardware}

Die Anwendung benötigt eine minimale Speicherkapazität von zwei Gigabyte.
Allerdings empfiehlt es sich, mehr Speicherplatz zur Verfügung zu stellen, um die wachsende Größe der Datenbank
zu unterstützen. \medskip

Außerdem ist zu beachten, dass das Backend keine Geräte mit einer ARM-Architektur unterstützt.
Dementsprechend sollte für die Ausführung ein Gerät mit einer anderen Architektur genutzt werden.

\subsection{Software}
\label{subsec:software}

Das Backend und das Frontend werden in zwei separaten Dockercontainern ausgeführt.
Dafür muss auf dem ausführendem Gerät Docker installiert sein.
Die Anwendung wurde entwickelt und getestet mit der Dockerversion 4.21.1;
auch wenn vermutlich ältere und neuere Versionen ebenfalls zu dem gewünschtem Ergebnis führen, können wir dies
nicht garantieren. \medskip

Damit der Webserver auch über eine öffentliche Adresse erreichbar ist, muss das ausführende
Gerät den Port weiterleiten. \medskip

Sollte eine Windows- oder Androidapplikation gebaut werden, fallen allerdings einige zusätzliche Forderungen an.
In diesem Fall muss auf dem Gerät Flutter installiert sein.
Hierbei wurde die Version 3.10.3 genutzt, die Unterstützung von älteren und neueren Versionen ist nicht garantiert.
Für die Erstellung der Windowsanwendung muss außerdem Windows installiert sein.

\section{Bauen der Anwendung}
\label{sec:bauen}

Das Backend und das Frontend kann getrennt gebaut werden.
Für das Frontend liegt eine Dockerfile für den Bau des Images vor.
Sollte das Backend unter einer anderen Adresse als localhost:8080 laufen, kann diese in der .env Datei
gesetzt werden.
Wird das Image für den Webserver ausgeführt, läuft der Service lokal im Container über den Port 80. \medskip

Auch für das Backend liegt eine Dockerfile vor.
Nachdem dieses ausgeführt wurde, läuft dieses lokal im Container und ist dort auf Port 8080 erreichbar. \medskip

Für die Erstellung der Windows- oder Androidanwendung muss zu dem Verzeichnis code/frontend/fastlane navigiert werden
und entweder der Befehl \enquote{flutter build windows -{-}release} oder \enquote{flutter build apk -{-}release} ausgeführt werden.
Daraufhin wird die .exe oder .apk erstellt und kann auf andere Geräte verteilt werden.
Wichtig ist hierbei nur, dass Adresse des Geräts auf dem das Backend läuft, eingegeben wurde.

\section{Ausführen der Anwendung}
\label{sec:ausfuehren}

Für die Zusammenschaltung der beiden Anwendungen wurde eine docker-compose.yml beigelegt,
welche bei Ausführung das Backend und das Frontend baut und startet.
Sobald diese gestartet sind, ist das Backend über den Port 8080 und das Frontend über den Port 8081 auf dem ausführenden
Gerät erreichbar.
Die Konfiguration des Backends kann über die backend.properties Datei innerhalb des config Ordners verändert werden.
Für die erste Darstellung liegen einige Beispieldaten vor, die bei dem Anstarten des Backends in die Datenbank
gelesen wird.
Sollten diese nicht erwünscht sein, kann der Wert von \enquote{load-demo-data} auf \enquote{false} gesetzt werden.
Nach erneutem Ausführen der docker-compose.yml tritt diese Änderung in Kraft.